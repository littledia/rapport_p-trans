\documentclass{article}

%------------- Préambule --------------
\usepackage[utf8]{inputenc}
\usepackage[francais]{babel}
\usepackage[T1]{fontenc}

\usepackage{graphicx} % Pour les figure
\graphicspath{{images/}} % Toutes les images se trouvent dans le dossier images
\usepackage{float}
\usepackage[T1]{tipa} % Pour la phonétique
\usepackage[francais]{minitoc} % Pour les mini sommaires

\usepackage{hyperref} % Pour les liens hypertext
\hypersetup{colorlinks,%
            citecolor=black,%
            filecolor=black,%
            linkcolor=black,%
            urlcolor=blue}


%------------- Corps --------------
\begin{document}
	
	%------------------------------------------------- Page de garde -------------------------------------------------%
\begin{titlepage} 
	\begin{center}
		\Huge{\textbf{Dimension Commerciale}}\\
		\vspace{0,5cm}
		\large {Doriane \bsc{Peresse}, Maxime \bsc{Catoire}}\\
		\vspace{0,5cm}
		\large {HES - \today}
	
		\vfill 

		\huge{Dans quelle mesure l'amélioration de la plateforme par notre projet peut-elle donner à l'entreprise l'opportunité d'étendre la distribution de son produit ?}

		\vfill 
	
	\end{center}
\end{titlepage}
	
	
	\tableofcontents
	
	\vspace{2cm}
		
	\begin{itemize}
		\item se situer sur une vue globale de l'application
		\item présenter en début de rapport le contexte d'utilisation et les finalités de l'application :
		\item activités concernées, profils des utilisateurs, fonctions
		\item
		\item étude de marché (demande, offre, environnement)
		\item stratégie commerciale (positionnement, mix marketing ...)
		\item Cette analyse doit permettre de formuler des propositions en matière de stratégie marketing et/ou des suggestions d'actions concrètes à mener pour un éventuel développement commercial de l'application.
	\end{itemize}
	
	\newpage
	
	%---------------------------------------- Introduction ----------------------------------------%
	%---------------------------------------- Introduction ----------------------------------------%


	
	\section{Introduction}
	Ce projet transversal est issu de la collaboration de l'Ecole Polytechnique de l'Université de Nantes et de l'entreprise Project2Cloud. Il sera mené de bout en bout par deux étudiants de quatrième année du cycle ingénieur du département informatique, supervisés par deux professeurs référents. Il s'étendra de septembre à mai. L'objectif est de permettre aux étudiants en charge de réaliser un projet professionnel dans le cadre de leurs études. Au delà de la dimension technique, les projets informatiques intègres des notions plus larges et plus actuelles. Il est donc important de prendre conscience que les projets transversaux présentent des enjeux organisationnels, marketing et/ou écologiques. Il conviendra de faire une étude d'une de ces trois dimensions, en s'appuyant sur les enseignements dispensés dans le module Homme Entreprise et Société. 
	
	Nous commencerons par présenter l'entreprise et son produit, puis nous détaillerons le projet qui nous a été proposé. Nous continuerons en présentant la problématique marketing que nous avons décidé de développer, ainsi que l'explication du choix de cette dimension. Enfin, nous ferons l'étude marketing de ce projet, et terminerons pas une conclusion qui synthétisera notre étude. 
	
	
	%---------------------------------------- L'entreprise et le produit ----------------------------------------%
	%---------------------------------------- L'entreprise et le produit ----------------------------------------%


	\section{Project2Cloud : l'entreprise et le produit}
	
	L'entreprise Project2Cloud a été créée par trois ingénieurs : Jean-Michel Maillet, Alexandre Langlais et Edouard Escallier.
	
	Project2Cloud est également le nom du projet que l'entreprise est en train de dévolopper. P2C (abréviation courante de Project2Cloud) est une plate-forme de gestion collaborative. Un site de présentation est déjà en ligne à l'adresse \href{http://project2cloud.org/}{http://project2cloud.org/}. Elle propose différents services de communication, tels que la vidéo-conférence (ou visio-conférence), le chat et la conversation audio. Le but est d'offrir aux utilisateurs un panel complet de communication, leur permettant d'échanger en temps réel des informations et des services. P2C innove également avec la possibilité d'associer aux sessions de dialogue des documents de n'importe quel types (texte, vidéos...). Le document associé à la session permet de créer une "mémoire" numérique du document, autour de laquelle de nouvelles connaissances peuvent venir s'ajouter à tout moment. On obtient ainsi une timeline (ligne de chronologique) sur laquelle est organisé tous les documents, vidéo, etc... d'un sujet. Il y a donc un réel suivi sur un fil de discussion.

		
	Cette plateforme s'adresse à des professionnels souhaitant conserver une trace de leur réunion et/ou visio-conférence. Cela est d'autant plus utile s'il y a des absents, puisque ces derniers peuvent savoir exactement ce qui a été dit en visionnant les vidéos.
		

	L'entreprise P2C  est géré par trois informaticiens passionnés depuis maintenant 1 an. La plate-forme P2C est pour le moment en béta-test privé, c'est à dire qu'elle n'est pas encore accessible au grand public. Seuls ont accès un nombre restreint d'entreprises volontaires et sélectionnées, qui envoient régulièrement des retours sur le produit. Pour les besoins du projet, un accès nous a été donné à la version opérationnelle de la plate-forme. Cette version nous permet d'avoir une idée de l'application qui sera prochainement disponible.
	
	
	%---------------------------------------- Notre projet ----------------------------------------%
	%---------------------------------------- Notre projet ----------------------------------------%


\section{Notre projet : Transcriptions audio vers texte pour des réunions enregistrées}
	Notre projet s'implantera directement dans la plate-forme P2C. Comme nous l'avons expliqué précédemment, P2C permettra aux utilisateurs de communiquer facilement grâce à un panel complet d'échanges vidéo, audio ou texte. Le but de notre projet est de faciliter l'indexation des conférences vidéos, c'est à dire de mieux séquencer temporellement les vidéos enregistrées et ainsi aider la recherche de vidéo par mots clés. Pour cela, nous devrons ajouter une fonctionnalité supplémentaire : la transcription vocale des conversations. Concrètement, lors d'une conférence vidéo, les conversations des différents participants seront traduites en texte. Ce même texte sera enregistré sur le serveur de la plateforme. Le but est de permettre à un utilisateur de retrouver un mot ou un groupe de mot prononcé pendant une réunion, et de pouvoir visionner la vidéo de cette dernière à l'instant où le mot a été prononcé. On créé ainsi une "mémoire" des réunions, avec la possibilité de retrouver des informations très précises à des moments clés. 
	
	
	Notre projet est donc de convertir les conversations en texte, d'enregistrer ces transcriptions, et de indexer ces conversations à partir de mots clés. L'indexation se fait dans un second fichier : lorsque le moteur de transcription reconnait un mot clé entré par l'utilisateur, il le met dans le fichier d'indexation avec une information sur le temps, c'est à dire lorsque le mot a été prononcé dans la réunion.
	
	
	La difficulté du projet est que le projet P2C est destiné à des entreprises françaises (dans un premier temps). Il faut donc que le moteur de transcriptions fonctionne avec du français, or à ce jour il n'existe pas de moteur performant dans cette langue.


	%---------------------------------------- Analyse marketing ----------------------------------------%
	%---------------------------------------- Analyse marketing ----------------------------------------%



\section{Analyse marketing notre projet}
	Notre projet vise donc à apporter une fonctionnalité supplémentaire à une plate-forme pré-existante. Cet aspect est intéressant, dans la mesure où des sites (ou logiciels) proposant des conférences vidéos existent déjà (comme Skype par exemple). Project2Cloud propose de nouvelles fonctionnalités encore peu, voire pas développées, dont notre module de transcription vocale. On est donc en droit de se demander dans quelle mesure l'amélioration de la plateforme par notre module peut-elle donner à l'entreprise l'opportunité d'étendre la distribution de son produit ?
	


	%---------------------------------------- Conclusion ----------------------------------------%
		%--------------------------------------------------------- Conclusion ------------------------------------------------------%

	\addstarredpart{Conclusion}
	\chapter*{Conclusion}
	
	Cette conclusion doit faire la synthèse de l'état d'avancement du projet, de votre avis par rapport à cet état d'avancement, et décrire brièvement les prochaines étapes du projet.


 

	


	
	
	
	
\end{document}