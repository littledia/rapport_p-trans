%---------------------------------------- Introduction ----------------------------------------%


	
	\section{Introduction}
	Ce projet transversal est issu de la collaboration de l'Ecole Polytechnique de l'Université de Nantes et de l'entreprise Project2Cloud. Il sera mené de bout en bout par deux étudiants de quatrième année du cycle ingénieur du département informatique, supervisés par deux professeurs référents. Il s'étendra de septembre à mai. L'objectif est de permettre aux étudiants en charge de réaliser un projet professionnel dans le cadre de leurs études. Au delà de la dimension technique, les projets informatiques intègres des notions plus larges et plus actuelles. Il est donc important de prendre conscience que les projets transversaux présentent des enjeux organisationnels, marketing et/ou écologiques. Il conviendra de faire une étude d'une de ces trois dimensions, en s'appuyant sur les enseignements dispensés dans le module Homme Entreprise et Société. 
	
	Nous commencerons par présenter l'entreprise et son produit, puis nous détaillerons le projet qui nous a été proposé. Nous continuerons en présentant la problématique marketing que nous avons décidé de développer, ainsi que l'explication du choix de cette dimension. Enfin, nous ferons l'étude marketing de ce projet, et terminerons pas une conclusion qui synthétisera notre étude. 