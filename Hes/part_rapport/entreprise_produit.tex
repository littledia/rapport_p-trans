%---------------------------------------- L'entreprise et le produit ----------------------------------------%


	\section{Project2Cloud : l'entreprise et le produit}
	
	L'entreprise Project2Cloud a été créée par trois ingénieurs : Jean-Michel Maillet, Alexandre Langlais et Edouard Escallier.
	
	Project2Cloud est également le nom du projet que l'entreprise est en train de dévolopper. P2C (abréviation courante de Project2Cloud) est une plate-forme de gestion collaborative. Un site de présentation est déjà en ligne à l'adresse \href{http://project2cloud.org/}{http://project2cloud.org/}. Elle propose différents services de communication, tels que la vidéo-conférence (ou visio-conférence), le chat et la conversation audio. Le but est d'offrir aux utilisateurs un panel complet de communication, leur permettant d'échanger en temps réel des informations et des services. P2C innove également avec la possibilité d'associer aux sessions de dialogue des documents de n'importe quel types (texte, vidéos...). Le document associé à la session permet de créer une "mémoire" numérique du document, autour de laquelle de nouvelles connaissances peuvent venir s'ajouter à tout moment. On obtient ainsi une timeline (ligne de chronologique) sur laquelle est organisé tous les documents, vidéo, etc... d'un sujet. Il y a donc un réel suivi sur un fil de discussion.

		
	Cette plateforme s'adresse à des professionnels souhaitant conserver une trace de leur réunion et/ou visio-conférence. Cela est d'autant plus utile s'il y a des absents, puisque ces derniers peuvent savoir exactement ce qui a été dit en visionnant les vidéos.
		

	L'entreprise P2C  est géré par trois informaticiens passionnés depuis maintenant 1 an. La plate-forme P2C est pour le moment en béta-test privé, c'est à dire qu'elle n'est pas encore accessible au grand public. Seuls ont accès un nombre restreint d'entreprises volontaires et sélectionnées, qui envoient régulièrement des retours sur le produit. Pour les besoins du projet, un accès nous a été donné à la version opérationnelle de la plate-forme. Cette version nous permet d'avoir une idée de l'application qui sera prochainement disponible.