	%------------------------------------------------- Procédure de recette -------------------------------------------------%

	\part{Procédure de recette}
	
	\setcounter{chapter}{1} % Pour recommencer la numérotation des chapitres à 1
	\setcounter{section}{0} % Pour recommencer la numérotation des section à 1
	
	\section{Introduction}
	Notre projet s'intégrera dans une plateforme déjà existante. Pour le tester, nous devrons vérifier qu'il répond aux attentes fonctionnelles, mais qu'il s'intègre également correctement dans la plateforme hôte. 
	Dans ce cahier de recette nous présenterons un batterie de tests visant à faire valider par le client l'intégralité des modules présentés dans le cahier des charges. Nous présenterons ici les tests sous la forme :
	\\
	\begin{center}
		Test effectué: \textit{Résultat(s) attendu(s)}
	\end{center}
		
	
	\section{Tests d'intégration du module dans la plateforme}


	Démarrage d'une vidéo conférence sur le site "Project2Cloud" : \textit{La vidéo conférence doit démarrer sans message d'erreur ni bug.}

	Lire un texte contenant les mots clés : "test", "réunion", "projet", "conférence" : \textit{La vidéo conférence se poursuit sans interruption ni ralentissement.}


	Couper la vidéo conférence : \textit{La vidéo conférence se termine sans erreur.}


	Ouvrir le module de recherche de texte : \textit{Une interface permettant de saisir du texte doit apparaitre}


	Saisir le mot "réunion" dans la zone de texte: \textit{La zone de texte doit se remplir du mot saisi}


	Valider la recherche en cliquant sur le bouton "Rechercher" : \textit{Une nouvelle fenêtre s'ouvre et affiche l'intégralité des occurrences du mot "réunion" trouvés.}
	
	