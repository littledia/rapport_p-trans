%------------------------------------------------- Réalisation -------------------------------------------------%
	
	\part{Réalisation}
	
	\setcounter{chapter}{0} % Pour recommencer la numérotation des chapitres à 1
	\setcounter{section}{0} % Pour recommencer la numérotation des section à 1
	
	\renewcommand*{\theHchapter}{\thepart.\thechapter}

Le rapport de réalisation est un rapport court mais qui comporte généralement des annexes longues.

Un premier chapitre décrit les outils utilisés pour la réalisation (IDE, serveur de version, logiciel de test, etc).

Un second chapitre montre le résultat du déroulement du logiciel généralement grâce à des snapshots ce qui doit prouver au lecteur ce qui fonctionne et ce qui ne fonctionne pas.

Un troisième chapitre explicite les spécificités qui ont été mises en oeuvre afin de répondre à la modélisation ou justement les écarts avec cette modélisation. Ce chapitre renvoie sur une annexe avec le code

Un quatrième chapitre explicite les tests prévus et donne un aperçu global de leurs résultats. Ce chapitre renvoie sur le document des tests détaillés (cf. chapitre suivant).

Une conclusion donne le résumé du projet et les perspectives à en tirer.

		\chapter{Outils utilisés}
		
		\section{IDE}
		Pour ce projet, nous avons utilisés trois langages de programmations différents :
		\begin{itemize}
			\item Java
			\item XML
			\item C
		\end{itemize}
		Différents IDE ont donc été utilisés.
		
		\subsection{Java}
		Pour le langage Java, nous avons utilisé la plateforme d'Eclipse. Celle-ci offre un large panel d'outils facilitant la programmation et la gestion d'erreur, rendant cet IDE un choix de prédilection.
		
		\subsection{XML}
		Le langage XML tel que nous l'avons utilisé n'étant pas très complexe, nous nous sommes contentés d'utiliser un simple éditeur de texte.
	
		\subsection{C} 
		Seul un programme a été codé en C. Nous avons utilisé l'IDE Xcode de Mac OS X pour ce programme, qui offre, de la même manière qu'Eclipse pour Java, de bon outils pour la programmation et une bonne gestion des erreurs.
		
		
		
		\section{Serveur de version}
		
		Un logiciel de gestion de version est un outils très puissant permettant de garder un historique des différentes modifications apportées à un fichier. Cela permet, lorsque des bugs persistent de trop après une modification, de revenir à une ancienne version qui fonctionne.
		
		Nous avons choisi d'utiliser le logiciel GitHub

		\chapter{Déroulement du projet}

		\chapter{Spécificités mises en oeuvres}

		\chapter{Tests prévus}

		\chapter{Conclusion}
		
		
		
		
		
		
		
		
		
		
		
		
		
		
		
		
		
		
		
		