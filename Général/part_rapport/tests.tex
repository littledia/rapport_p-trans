%------------------------------------------------- Tests -------------------------------------------------%
	
	\part{Tests}
	
	\setcounter{chapter}{0} % Pour recommencer la numérotation des chapitres à 1
	\setcounter{section}{0} % Pour recommencer la numérotation des section à 1
	
	\renewcommand*{\theHchapter}{\thepart.\thechapter}
	\chapter{a}
	
	Ce dossier de tests doit rendre compte de la vérification du logiciel (test unitaire, test d'intégration, recette interne)

Test unitaire :

Test d'intégration : 

Recette interne : Avant de livrer le travail au client, vous devrez effectuer vous-même la procédure de recette "en interne". Pour chaque test décrit, vous noterez le résultat obtenu, ainsi que sa conformité par rapport au résultat attendu. Lorsqu'une différence est constatée, vous la commenterez (erreur inattendue, erreur connue mais pas prioritaire, coût présumé de la mise en conformité...).