	%------------------------------------------------- Contexte du projet ------------------------------------------------%

	\part{Contexte du projet}
	
	\setcounter{chapter}{1} % Pour recommencer la numérotation des chapitres à 1
	\setcounter{section}{0} % Pour recommencer la numérotation des section à 1
	
	\section{L'entreprise : Project2Cloud}
	Project2Cloud est une plate-forme de gestion collaborative. Une site de présentation est déjà en ligne à l'adresse http://project2cloud.org/. Elle propose différents services de communication, tels que la vidéo-conférence, le chat et la conversation audio. Le but est d'offrir aux utilisateurs un panel complet de communication, leur permettant d'échanger en temps réel des informations et des services. Project2Cloud innove également avec la possibilité d'associer aux sessions de dialogue des documents de différents types (texte, vidéos...). Le document associé à la session permet de créer une "mémoire" numérique du document, autour de laquelle de nouvelles connaissances peuvent venir s'ajouter à tout moment.
	
	P2C (abréviation courante de Project2Cloud) est géré par trois informaticiens passionnés depuis maintenant 1 an. La plate-forme P2C est pour le moment en béta-test privé, c'est à dire qu'elle n'est pas encore accessible. Elle est actuellement testée par plusieurs industriels. Pour les besoins du projet, un accès nous a été donné à la version opérationnelle de la plate-forme. Cette version nous permet d'avoir une idée de l'application qui sera prochainement disponible.
	
	\section{Notre projet : Transcriptions audio vers texte pour des réunions enregistrées}
	Notre projet s'implantera directement dans la plate-forme P2C. Comme nous l'avons expliqué précédemment, P2C permettra aux utilisateurs de communiquer facilement grâce à un panel complet d'échanges vidéo, audio ou texte. Le but de notre projet est de faciliter l'indexation des conférences vidéos (c'est à dire de mieux séquencer temporellement les vidéos enregistrées). Pour cela, nous devrons ajouter une fonctionnalité supplémentaire : la transcription vocale des conversations. Concrètement, lors d'une conférence vidéo, les conversations des différents participants seront traduites en texte. Ce même texte sera enregistré sur un serveur. Le but est de permettre à un utilisateur de retrouver un mot ou un groupe de mot prononcé pendant une conférence, et de pouvoir visionner cette dernière à l'instant où le mot a été prononcé. On créé ainsi une "mémoire" des conférences, avec la possibilité de retrouver des informations très précises à des moments clés. 
	Notre projet est donc de convertir les conversations en texte, de les enregistrer, et de pouvoir indexer ces conversations à partir de mots clés.